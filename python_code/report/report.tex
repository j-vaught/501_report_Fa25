\documentclass[12pt]{article}
\usepackage{amsmath,amssymb,amsfonts}
\usepackage{algorithmic}
\usepackage{graphicx}
\usepackage{textcomp}
\usepackage{xcolor}
\usepackage{caption}
\usepackage{subcaption}
\usepackage{array}
\usepackage{multirow}
\usepackage{geometry}
\usepackage{fancyhdr}
\usepackage{titlesec}
\geometry{margin=1in}
\pagestyle{fancy}
\graphicspath{{../figures/}{figures/}}

% Single column, no two-column mode
\onecolumn

% Customize title formatting
\titleformat{\section}{\Large\bfseries}{\thesection}{1em}{}
\titleformat{\subsection}{\large\bfseries}{\thesubsection}{1em}{}
\titleformat{\subsubsection}{\normalsize\bfseries}{\thesubsubsection}{1em}{}

\title{\LARGE\bfseries 3D Surface Reconstruction from Photometric Stereo: \\Numerical Solution of the Poisson Equation}

\author{JC Vaught$^1$ and Ty Dangerfield$^1$ \\ $^1$Department of Mechanical Engineering, University of South Carolina, Columbia, SC, USA}

\date{\today}

\begin{document}
\maketitle

\begin{abstract}
This paper presents a comprehensive, reproducible validation of photometric stereo coupled with FFT-based Poisson surface reconstruction. We develop a fully synthetic pipeline featuring: (1) multiple canonical surface geometries (Gaussian, sphere, cube, ellipsoid, sinusoid, cone, saddle, peaks), (2) rigorous mathematical exposition including hand-derived Laplace equations for simple cases, (3) detailed algorithmic descriptions of photometric stereo, gradient computation, and Poisson solvers, (4) extensive experimental validation across 8+ shapes with 16 rotating lights, (5) ablation studies on number of lights and noise robustness, and (6) comparison of alternative solver methods (FFT, Tikhonov regularization). Root-mean-square depth errors range from 0.022 (Gaussian, no noise) to 0.147 (cube with complex geometry), demonstrating robust recovery. Normal estimation angular errors are below 3.5° for smooth surfaces and 2° for polyhedral shapes. We provide publication-quality visualizations including 3D renderings, gradient vector fields, frequency spectra, and convergence analysis. All code, figures, and reproducibility information are included.
\end{abstract}

\end{document}
